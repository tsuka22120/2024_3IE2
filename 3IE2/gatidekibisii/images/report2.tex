\documentclass[a4j]{jarticle}

% 数式
\usepackage{amsmath,amsfonts}
\usepackage{bm}
% 画像
\usepackage[dvipdfmx]{graphicx}
\usepackage{listings,jvlisting}
\usepackage{jlisting}


\lstset{
basicstyle={\ttfamily},
identifierstyle={\small},
commentstyle={\smallitshape},
keywordstyle={\small\bfseries},
ndkeywordstyle={\small},
stringstyle={\small\ttfamily},
frame={tb},
breaklines=true,
columns=[l]{fullflexible},
numbers=left,
xrightmargin=0zw,
xleftmargin=3zw,
numberstyle={\scriptsize},
stepnumber=1,
numbersep=1zw,
lineskip=-0.5ex
}

% プログラミングリストコマンド
\renewcommand{\lstlistingname}{リスト}

% 画像挿入コマンド
\newcommand{\Figure}[4]{
\begin{figure}[H]
\centering
\includegraphics[width=#1\linewidth]{./images/#2}
\caption{#3}
\label{fig:#4}
\end{figure}
}
\begin{document}

\title{減算基板のリバースエンジニアリング}
\author{学籍番号:22120 \\ 組番号:222 \\名前:塚田 勇人}
\date{\today}
\maketitle

\newpage
\tableofcontents
\newpage

\section{目的}
電子天秤の作成のために,減算基板のリバースエンジニアリングを行い,基板の回路図を作成する.
リバースエンジニアリングとは,製品の作業工程の反対をたどって,製品の構造や仕組みについて考えることである.
そのために,減算基板の動作を確認して,マルチテスタを用いて,実際の回路を探査する.
そして,探査して得られた情報をもとに,回路図を作成する.
その過程を通して,減算基板についての理解を深めることが目的である.

\section{原理}

\section{実験環境}
本実験では,減算基板のリバースエンジニアリングを行う.
減算基板に関する基本的な知識や使われている電子部品について説明する.

\subsection{減算基板} \label{subsec:genzan}
減算基板は,複数の入力信号から1つの信号を選択して出力する基板である.
真理値表を表\ref{tab:selector}に示す.

\begin{table}[H]
  \caption{減算基板の真理値表}
  \centering
  \begin{tabular}{|cc|cc|}
    \hline
    SW0 & SW1 & OUT                     & EN \\
    \hline
    0   & 0   & \ast \textreferencemark & 0  \\
    0   & 1   & B                       & 1  \\
    1   & 0   & C                       & 1  \\
    1   & 1   & A                       & 1  \\
    \hline
  \end{tabular}
  \label{tab:selector}
\end{table}

\subsection{74LSシリーズIC}
今回の実験で用いるICのそれぞれの機能やピンアサインについて説明する.
本実験では,74LSシリーズのICを用いる.
74LSシリーズは,バイポーラトランジスタを用いて構成されていて,足が14本のICは,7番ピンがGND,14番ピンがVCCに接続されている.

\subsubsection{SN74LS04N}
SN74LS04Nとは,NOTゲートを6つ内蔵したICである.NOTゲートは,入力された信号を反転させる回路である.
SN74LS04Nのピンアサインを図\ref{fig:NOT}に示す.
\Figure{0.4}{SN74LS04N}{SN74LS04Nのピンアサイン}{NOT}

\subsubsection{SN74LS08N}
SN74LS08Nとは,ANDゲートを4つ内蔵したICである.ANDゲートは,入力された信号がすべてHighのときにHighを出力する回路である.
SN74LS08Nのピンアサインを図\ref{fig:AND}に示す.
\Figure{0.4}{SN74LS08N}{SN74LS08Nのピンアサイン}{AND}

\subsubsection{SN74LS32N}
SN74LS32Nとは,ORゲートを4つ内蔵したICである.ORゲートは,入力された信号のうち1つでもHighがあればHighを出力する回路である.
SN74LS32Nのピンアサインを図\ref{fig:OR}に示す.
\Figure{0.4}{SN74LS32N}{SN74LS32Nのピンアサイン}{OR}

\subsubsection{SN74LS86N}
SN74LS86Nとは,XORゲートを4つ内蔵したICである.XORゲートは,入力された信号が異なるときにHighを出力する回路である.
SN74LS86Nのピンアサインを図\ref{fig:XOR}に示す.
\Figure{0.4}{SN74LS86N}{SN74LS86Nのピンアサイン}{XOR}


\section{プログラムの設計と説明}

\section{プログラム}

\section{実行結果}

\section{考察}

\end{document}