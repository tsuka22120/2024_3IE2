\documentclass[a4j]{jarticle}

% 数式
\usepackage{amsmath,amsfonts}
\usepackage{bm}
% 画像
\usepackage[dvipdfmx]{graphicx}
\usepackage{listings,jvlisting}
\usepackage{jlisting}


\lstset{
basicstyle={\ttfamily},
identifierstyle={\small},
commentstyle={\smallitshape},
keywordstyle={\small\bfseries},
ndkeywordstyle={\small},
stringstyle={\small\ttfamily},
frame={tb},
breaklines=true,
columns=[l]{fullflexible},
numbers=left,
xrightmargin=0zw,
xleftmargin=3zw,
numberstyle={\scriptsize},
stepnumber=1,
numbersep=1zw,
lineskip=-0.5ex
}
\renewcommand{\lstlistingname}{リスト}

\begin{document}

\title{キューの実装}
\author{学籍番号:22120 \\ 組番号:222 \\名前:塚田 勇人}
\date{\today}
\maketitle

\newpage
\tableofcontents
\newpage

\section{目的}
与えられた課題をこなすことで、キューに対する理解を深めることをこのレポートの目的と
します。

\section{原理}
この章ではキューについてと、再構築するキュー、リングキュー、リストキューの三種類の
キューの原理について説明します。
\subsection{キュー}
キューは、データを一時的に保存するためのデータ構造の一つです。キューは、データを
先入れ先出し(FIFOと呼ばれる。First-In-First-Outの略)の原則に基づいて動作します。
キューは、データを追加するエンキューとデータを取り出すデキューの二つの操作を持ちます。
キューはデータの一時的な保存に使われるため、データの追加と取り出しの操作が頻繁に行われる
場合に適しています。

\subsection{再構築するキュー}
再構築するキューは、キューの要素を配列で表現したものです。再構築するキューは、
配列の個数を示す変数を持ち、エンキューとデキューの操作を行う際に、
配列の要素を移動させることでキューを実装します。再構築するキューは、
エンキューとデキューの操作が頻繁に行われる場合には、リングキューやリストキューに比べて
遅いという欠点があります。

\subsection{リングキュー}
リングキューは、再構築するキューの欠点を改善したキューです。配列の末尾と先頭が
つながっていると捉えることで、エンキューとデキューの操作を行う際に要素の移動を
行う必要がなくなり、キューのために用意された領域全体を使うことができます。
リングキューは、エンキューとデキューの操作が頻繁に行われる場合に適しています。

\subsection{リストキュー}
リストキューは上記の二つのキューと違い、連結リストを用いて動的にデータ領域を確保できる
キューです。動的にデータ領域を確保できるため、リングキューや再構築するキューに比べて
メモリの使用効率が良いという利点があります。リストキューは、データの個数が不定な場合に
適しています。

\section{実験環境}
キューの課題のプログラムを実装した環境を表\ref{table:environment}に示します。
\begin{table}
  \centering
  \caption{実験環境}
  \label{table:environment}
  \begin{tabular}{|c|c|}
    \hline
    OS    & Windows 11 Pro     \\
    \hline
    CPU   & AMD Ryzen 7 5800H  \\
    \hline
    メモリ   & 16GB               \\
    \hline
    コンパイラ & gcc version 11.4.0 \\
    \hline
    エディタ  & Visual Studio Code \\
    \hline
  \end{tabular}
\end{table}

\section{プログラムの設計と説明}
この章では、再構築するキュー、リングキュー、リストキューのプログラムの実装に使われている
データ構造や関数について説明します。

\subsection{再構築するキュー}
再構築するキューのプログラムでは、リスト\ref{queue}のようにキューを定義しています。
\begin{lstlisting}[caption=再構築するキュープログラムのキューの定義,label=queue]
  #define QUEUE_SIZE 4

  int queue[QUEUE_SIZE]; // キュー用の配列
  int quantity = 0; // キューに格納されているデータの個数
\end{lstlisting}


このプログラムでは表\ref{queuefunc}の関数を実装しています。
\begin{table}[htbp]
  \centering
  \caption{再構築するキューの関数}
  \label{queuefunc}
  \begin{tabular}{|c|c|}
    \hline
    関数名        & 説明                                 \\
    \hline
    enqueue    & キューにデータを積み上げる関数                    \\
    \hline
    dequeue    & キューからデータを取り出す関数                    \\
    \hline
    initQueue  & キュー用の配列を0で初期化する関数                  \\
    \hline
    showQueue  & キューの中身を表示する関数                      \\
    \hline
    showResult & 変数resultの値に応じて、対応するエラーメッセージを表示する関数 \\
    \hline
  \end{tabular}
\end{table}
それぞれの関数について説明します。

\subsubsection{enqueue関数}
enqueue関数は表\ref{table:enqueue}のように定義され、
リスト\ref{list:enqueue}のように実装されています。

\newpage

\begin{table}[htbp]
  \centering
  \caption{再構築するキューのenqueue関数}
  \label{table:enqueue}
  \begin{tabular}{|c|c|}
    \hline
    関数名 & enqueue              \\
    \hline
    型   & int                  \\
    \hline
    引数  & int data:キューに格納するデータ \\
    \hline
    戻り値 & 実行の成否を返す             \\
    \hline
    概要  & キューにデータを積み上げる関数      \\
    \hline
  \end{tabular}
\end{table}

\begin{lstlisting}[caption=再構築するキューのenqueue関数の実装,label=list:enqueue]
    int enqueue(int data) {
    // 残り領域があるか確認する
    if (quantity >= QUEUE_SIZE) {
        // 残り領域がない場合はエラーコードを返す
        return -101;
    }
    // データが自然数か確認する
    if (data <= 0) {
        // 自然数でない場合はエラーコードを返す
        return -102;
    }
    // 配列のキューにデータを保存する
    queue[quantity] = data;
    // キューのポインタをインクリメントする
    quantity++;
    // 返り値を返す
    return -100;
    }
\end{lstlisting}
enqueue関数は、次のような処理が実装されています。
\begin{enumerate}
  \item 残り領域があるか確認し、なかったらエラーコードを返す。
  \item データが自然数か確認し、自然数でなかったらエラーコードを返す。
  \item 配列のキューにデータを保存する。
  \item キューに格納されているデータの個数を増やす。
  \item 成功を示す返り値を返す。
\end{enumerate}

\subsubsection{dequeue関数}
dequeue関数は表\ref{table:dequeue}のように定義され、
リスト\ref{list:dequeue}のように実装されています。

\newpage

\begin{table}[htbp]
  \centering
  \caption{再構築するキューのdequeue関数}
  \label{table:dequeue}
  \begin{tabular}{|c|c|}
    \hline
    関数名 & dequeue             \\
    \hline
    型   & int                 \\
    \hline
    引数  & なし                  \\
    \hline
    戻り値 & 取り出したデータ、もしくはエラーコード \\
    \hline
    概要  & キューからデータを取り出す関数     \\
    \hline
  \end{tabular}
\end{table}

\begin{lstlisting}[caption=再構築するキューのdequeue関数の実装,label=list:dequeue]
    int dequeue(void) {
    // データが存在するかどうか確認する.
    if (quantity == 0) {
        // データが存在しない場合はエラーコードを返す.
        return -201;
    }
    // キューからデータをとりだす.
    int data = queue[0];
    // データの個数カウントを減らす
    quantity--;
    // 取り出されたデータ部分を埋めるように再構築する
    for (int i = 0; i < quantity; i++) {
        queue[i] = queue[i + 1];
    }
    queue[quantity] = 0;
    // データを返す
    return data;
    }
\end{lstlisting}

dequeue関数は、次のような処理が実装されています。
\begin{enumerate}
  \item データが存在するかどうか確認し、存在しない場合はエラーコードを返す。
  \item キューからデータを取り出す。
  \item キューに格納されているデータの個数を減らす。
  \item 取り出されたデータ部分を埋めるように再構築する。
  \item 取り出されたデータを返す。
\end{enumerate}

\subsubsection{initQueue関数}
initQueue関数は表\ref{table:initQueue}のように定義され、
リスト\ref{list:initQueue}のように実装されています。

\newpage

\begin{table}[htbp]
  \centering
  \caption{再構築するキューのinitQueue関数}
  \label{table:initQueue}
  \begin{tabular}{|c|c|}
    \hline
    関数名 & initQueue         \\
    \hline
    型   & int               \\
    \hline
    引数  & なし                \\
    \hline
    戻り値 & 成功したら0を返す         \\
    \hline
    概要  & キュー用の配列を0で初期化する関数 \\
    \hline
  \end{tabular}
\end{table}

\begin{lstlisting}[caption=再構築するキューのinitQueue関数の実装,label=list:initQueue]
    int initQueue() {
    // キューのデータを入れる配列をすべて 0 に初期化する.
    for (int i = 0; i < QUEUE_SIZE; i++) {
        queue[i] = 0;
    }
    // 格納データ個数を 0 に初期化する.
    quantity = 0;
    return 0;
    }
\end{lstlisting}

initQueue関数は、次のような処理が実装されています。
\begin{enumerate}
  \item キューのデータを入れる配列をすべて0に初期化する。
  \item 格納データ個数を0に初期化する。
  \item 成功を示す返り値を返す。
\end{enumerate}

\subsubsection{showQueue関数}
showQueue関数は表\ref{table:showQueue}のように定義され、
リスト\ref{list:showQueue}のように実装されています。
\begin{table}[htbp]
  \centering
  \caption{再構築するキューのshowQueue関数}
  \label{table:showQueue}
  \begin{tabular}{|c|c|}
    \hline
    関数名 & showQueue     \\
    \hline
    型   & int           \\
    \hline
    引数  & なし            \\
    \hline
    戻り値 & 成功したら0を返す     \\
    \hline
    概要  & キューの中身を表示する関数 \\
    \hline
  \end{tabular}
\end{table}

\begin{lstlisting}[caption=再構築するキューのshowQueue関数の実装,label=list:showQueue]
  int showQueue() {
    // 配列全体のデータを順に表示する.
    printf("|");
    for (int i = 0; i < QUEUE_SIZE; i++) {
        printf("%d", queue[i]);
        // データとデータの間に区切り文字「|」を表示する.
        printf("|");
      }
    return 0;
    }
\end{lstlisting}

showQueue関数は、次のような処理が実装されています。
\begin{enumerate}
  \item データとデータの間に区切り文字「|」を表示する。
  \item 配列全体のデータを順に表示する。
  \item 成功を示す返り値を返す。
\end{enumerate}

\subsubsection{showResult関数}
showResult関数は表\ref{table:showResult}のように定義され、
リスト\ref{list:showResult}のように実装されています。
\begin{table}[htbp]
  \centering
  \caption{再構築するキューのshowResult関数}
  \label{table:showResult}
  \begin{tabular}{|c|c|}
    \hline
    関数名 & showResult                         \\
    \hline
    型   & void                               \\
    \hline
    引数  & int result:エラーコード                  \\
    \hline
    戻り値 & なし                                 \\
    \hline
    概要  & 変数resultの値に応じて、対応するエラーメッセージを表示する関数 \\
    \hline
  \end{tabular}
\end{table}

\begin{lstlisting}[caption=再構築するキューのshowResult関数の実装,label=list:showResult]
  void showResult(int result) {
      // result の値に応じて,対応するエラーメッセージを表示する.
      switch (result) {
          case -100:
              printf(":Enqueue success\n");
              break;
          case -101:
              printf(":Enqueue error:No space left(ErrorCode-101)\n");
              break;
          case -102:
              printf(":Enqueue error:Not natural number was entered:(ErrorCode-102)\n");
              break;
          case -201:
              printf(":Dequeue error:Nothing data(ErrorCode-201)\n");
              break;
          case 0:
              printf(":success\n");
              break;
          default:
              printf(":Dequeue success\n");
              break;
      }
  }
\end{lstlisting}

showResult関数は、次のような処理が実装されています。
\begin{enumerate}
  \item resultの値を、次の項目に分けてエラーメッセージを表示する。
        \begin{itemize}
          \item -100:エンキュー成功
          \item -101:エンキューエラー:残り領域なし
          \item -102:エンキューエラー:自然数が入力されていない
          \item -201:デキューエラー:データがない
          \item 0:成功
          \item その他:デキュー成功
        \end{itemize}
\end{enumerate}

\subsection{リングキュー}
リングキューのプログラムでは、リスト\ref{ringqueue}のようにキューを定義しています。
\begin{lstlisting}[caption=リングキュープログラムのキューの定義,label=ringqueue]
  #define QUEUE_SIZE 5

  typedef struct {
      int data[QUEUE_SIZE];  // データが入る配列
      int wp;                // 次にデータを入れる場所の配列番号
      int quantity;          // データの個数
  } Queue;
\end{lstlisting}
リングキューでは表\ref{ringqueuefunc}の関数を実装しています。
再構築するキューでも実装している関数と同じ関数が実装されて
いますが、関数の内容は変わっています。

\begin{table}[htbp]
  \centering
  \caption{リングキューの関数}
  \label{ringqueuefunc}
  \begin{tabular}{|c|c|}
    \hline
    関数名          & 説明                                 \\
    \hline
    enqueue      & キューにデータを積み上げる関数                    \\
    \hline
    dequeue      & キューからデータを取り出す関数                    \\
    \hline
    initQueue    & キュー用の配列を0で初期化する関数                  \\
    \hline
    showQueue    & キューの中身を表示する関数                      \\
    \hline
    showResult   & 変数resultの値に応じて、対応するエラーメッセージを表示する関数 \\
    \hline
    dequeueIndex & キューからデータを取り出す際のインデックスを返す関数         \\
    \hline
  \end{tabular}
\end{table}
それぞれの関数について説明します。

\subsubsection{enqueue関数}
enqueue関数は表\ref{table:ringenqueue}のように定義され、
リスト\ref{list:ringenqueue}のように実装されています。

\newpage

\begin{table}[htbp]
  \centering
  \caption{リングキューのenqueue関数}
  \label{table:ringenqueue}
  \begin{tabular}{|c|c|}
    \hline
    関数名 & enqueue                         \\
    \hline
    型   & int                             \\
    \hline
    引数1 & Queue *obj:データを追加するキューの構造体のアドレス \\
    \hline
    引数2 & int data:キューに格納するデータ            \\
    \hline
    戻り値 & 実行の成否を返す                        \\
    \hline
    概要  & キューにデータを積み上げる関数                 \\
    \hline
  \end{tabular}
\end{table}

\begin{lstlisting}[caption=リングキューのenqueue関数の実装,label=list:ringenqueue]
  int enqueue(Queue *obj, int data) {
      // 残り領域があるか確認する
      if (obj->quantity >= QUEUE_SIZE) {
          // 残り領域がない場合はエラーコードを返す
          return -101;
      }
      // データが自然数か確認する
      if (data <= 0) {
          // 自然数でない場合はエラーコードを返す
          return -102;
      }
      // 配列のキューにデータを保存する
      obj->data[obj->wp] = data;
      // キューのポインタをインクリメントする
      obj->quantity++;
      if (obj->wp == QUEUE_SIZE - 1) {
          obj->wp = 0;
      } else {
          obj->wp++;
      }
      // 返り値を返す
      return -100;
  }
\end{lstlisting}

enqueue関数は、次のような処理が実装されています。
\begin{enumerate}
  \item 残り領域があるか確認し、なかったらエラーコードを返す。
  \item データが自然数か確認し、自然数でなかったらエラーコードを返す。
  \item 配列のキューにデータを保存する。
  \item キューに格納されているデータの個数を増やす。
  \item データが配列の最後まで入ったら次のデータを入れる配列番号を0とし、そうでない場合はインクリメントする。
  \item 成功を示す返り値を返す。
\end{enumerate}

\subsubsection{dequeue関数}
dequeue関数は表\ref{table:ringdequeue}のように定義され、
リスト\ref{list:ringdequeue}のように実装されています。

\begin{table}[htbp]
  \centering
  \caption{リングキューのdequeue関数}
  \label{table:ringdequeue}
  \begin{tabular}{|c|c|}
    \hline
    関数名 & dequeue                         \\
    \hline
    型   & int                             \\
    \hline
    引数  & Queue *obj:データを取り出すキューの構造体のアドレス \\
    \hline
    戻り値 & 取り出したデータ、もしくはエラーコード             \\
    \hline
    概要  & キューからデータを取り出す関数                 \\
    \hline
  \end{tabular}
\end{table}

\begin{lstlisting}[caption=リングキューのdequeue関数の実装,label=list:ringdequeue]
  int dequeue(Queue *obj) {
      int index;
      index = dequeueIndex(*obj);
      // データが存在するかどうか確認する.
      if (obj->quantity == 0) {
          // データが存在しない場合はエラーコードを返す.
          return -201;
      }
      // キューからデータをとりだす.
      int data = obj->data[index];
      // 取り出されたデータ部分を埋めるように再構築する
      obj->data[index] = 0;
      // データの個数のカウントを減らす.
      obj->quantity--;
      // データを返す.
      return data;
  }
\end{lstlisting}

dequeue関数は、次のような処理が実装されています。
\begin{enumerate}
  \item キューからデータを取り出す際のインデックスを取得する。
  \item データが存在するかどうか確認し、存在しない場合はエラーコードを返す。
  \item キューからデータを取り出す。
  \item 取り出されたデータ部分を埋めるように再構築する。
  \item キューに格納されているデータの個数を減らす。
  \item 取り出されたデータを返す。
\end{enumerate}

\subsubsection{initQueue関数}
initQueue関数は表\ref{table:ringinitQueue}のように定義され、
リスト\ref{list:ringinitQueue}のように実装されています。
\begin{table}[htbp]
  \centering
  \caption{リングキューのinitQueue関数}
  \label{table:ringinitQueue}
  \begin{tabular}{|c|c|}
    \hline
    関数名 & initQueue                    \\
    \hline
    型   & int                          \\
    \hline
    引数  & Queue *obj:初期化するキューの構造体のアドレス \\
    \hline
    戻り値 & 成功したら-400を返す                 \\
    \hline
    概要  & キュー用の配列を0で初期化する関数            \\
    \hline
  \end{tabular}
\end{table}

\begin{lstlisting}[caption=リングキューのinitQueue関数の実装,label=list:ringinitQueue]
  int initQueue(Queue *obj) {
      int i;
      // キューのデータを入れる配列をすべて 0 に初期化する.
      for (i = 0; i < QUEUE_SIZE; i++) {
          obj->data[i] = 0;
      }
      // キューのデータ格納個数を 0 に初期化する
      obj->quantity = 0;
      // キューの wp ポインタを 0 に初期化する.
      obj->wp = 0;
      return -400;
  }
\end{lstlisting}

initQueue関数は、次のような処理が実装されています。
\begin{enumerate}
  \item キューのデータを入れる配列をすべて0に初期化する。
  \item キューのデータ格納個数を0に初期化する。
  \item 次のデータを入れる配列番号を0に初期化する。
  \item 成功を示す返り値を返す。
\end{enumerate}

\subsubsection{showQueue関数}
showQueue関数は表\ref{table:showQueue}のように定義され、
リスト\ref{list:showQueue}のように実装されています。

\begin{table}[htbp]
  \centering
  \caption{リングキューのshowQueue関数}
  \label{table:showQueue}
  \begin{tabular}{|c|c|}
    \hline
    関数名 & showQueue                   \\
    \hline
    型   & int                         \\
    \hline
    引数  & Queue *obj:表示するキューの構造体のアドレス \\
    \hline
    戻り値 & 成功したら-500を返す                \\
    \hline
    概要  & キューのデータを表示する関数              \\
    \hline
  \end{tabular}
\end{table}

\begin{lstlisting}[caption=リングキューのshowQueue関数の実装,label=list:showQueue]
  int showQueue(Queue *obj) {
      int i;
      // 配列全体のデータを順に表示する.
      // データとデータの間に区切り文字「|」を表示する.
      printf("|");
      for (i = 0; i < QUEUE_SIZE; i++) {
          printf("%d", obj->data[i]);
          printf("|");
      }
      return -500;
  }
\end{lstlisting}

showQueue関数は、次のような処理が実装されています。
\begin{enumerate}
  \item データとデータの間に区切り文字「|」を表示する。
  \item 配列全体のデータを順に表示する。
  \item 成功を示す返り値を返す。
\end{enumerate}

\subsubsection{showResult関数}
showResult関数は表\ref{table:showResult}のように定義され、
リスト\ref{list:showResult}のように実装されています。

\begin{table}[htbp]
  \centering
  \caption{リングキューのshowResult関数}
  \label{table:showResult}
  \begin{tabular}{|c|c|}
    \hline
    関数名 & showResult                         \\
    \hline
    型   & void                               \\
    \hline
    引数  & int result:エラーコード                  \\
    \hline
    戻り値 & なし                                 \\
    \hline
    概要  & 変数resultの値に応じて、対応するエラーメッセージを表示する関数 \\
    \hline
  \end{tabular}
\end{table}

\begin{lstlisting}[caption=リングキューのshowResult関数の実装,label=list:showResult]
  void showResult(int result) {
      switch (result) {
          case -100:
              printf(":Enqueue success\n");
              break;
          case -101:
              printf(":Enqueue error:No space left(ErrorCode-101)\n");
              break;
          case -102:
              printf(":Enqueue error:Not natural number was entered:(ErrorCode-102)\n");
              break;
          case -201:
              printf(":Dequeue error:Nothing data(ErrorCode-201)\n");
              break;
          case -400:
              printf(":Init success\n");
              break;
          case -500:
              printf(":Print success\n");
              break;
          default:
              printf(":Dequeue success\n");
              break;
      }
      // result の値に応じて,対応するエラーメッセージを表示する.
  }
\end{lstlisting}

showResult関数は、次のような処理が実装されています。
\begin{enumerate}
  \item resultの値を、次の項目に分けてエラーメッセージを表示する。
        \begin{itemize}
          \item -100:エンキュー成功
          \item -101:エンキューエラー:残り領域なし
          \item -102:エンキューエラー:自然数が入力されていない
          \item -201:デキューエラー:データがない
          \item -400:初期化成功
          \item -500:表示成功
          \item その他:デキュー成功
        \end{itemize}
\end{enumerate}

\subsubsection{dequeueIndex関数}
dequeueIndex関数は表\ref{table:dequeueIndex}のように定義され、
リスト\ref{list:dequeueIndex}のように実装されています。

\begin{table}[htbp]
  \centering
  \caption{リングキューのdequeueIndex関数}
  \label{table:dequeueIndex}
  \begin{tabular}{|c|c|}
    \hline
    関数名 & dequeueIndex               \\
    \hline
    型   & int                        \\
    \hline
    引数  & Queue obj:データを取り出すキューの構造体  \\
    \hline
    戻り値 & キューからデータを取り出す際のインデックス      \\
    \hline
    概要  & キューからデータを取り出す際のインデックスを返す関数 \\
    \hline
  \end{tabular}
\end{table}

\begin{lstlisting}[caption=リングキューのdequeueIndex関数の実装,label=list:dequeueIndex]
  int dequeueIndex(Queue obj) {
    if (obj.wp - obj.quantity < 0) {
        return QUEUE_SIZE + obj.wp - obj.quantity;
    } else {
        return obj.wp - obj.quantity;
    }
}
\end{lstlisting}

dequeueIndex関数は、次のような処理が実装されています。
\begin{enumerate}
  \item 次のデータを入れる配列番号からデータの個数を引いた値が0未満の場合、
        次のデータを入れる配列番号から、配列のサイズを足してデータの個数を引いた値を返す。
        そうでない場合は次のデータを入れる配列番号から配列のデータの個数を引いた値を返す。
\end{enumerate}

\subsection{リストキュー}
リストキューのプログラムでは、リスト\ref{listqueue}のようにキューを定義しています。
\begin{lstlisting}[caption=リストキュープログラムのキューの定義,label=listqueue]
  struct queue {
    int value;
    struct queue *addr;
  };

  struct queue *bottom_queue =
  NULL;  // 最古のキューのアドレスを記憶しておくポインタ

  struct queue *top_queue =
  NULL;  // 最新のキューのアドレスを記憶しておくポインタ
\end{lstlisting}

リストキューでは表\ref{listqueuefunc}の関数を実装しています。
再構築するキューやリングキューでも実装している関数と同じ関数が
実装されていますが、関数の内容は変わっています。

\begin{table}[htbp]
  \centering
  \caption{リストキューの関数}
  \label{listqueuefunc}
  \begin{tabular}{|c|c|}
    \hline
    関数名        & 説明                                 \\
    \hline
    enqueue    & キューにデータを積み上げる関数                    \\
    \hline
    dequeue    & キューからデータを取り出す関数                    \\
    \hline
    showQueue  & キューの中身を表示する関数                      \\
    \hline
    showResult & 変数resultの値に応じて、対応するエラーメッセージを表示する関数 \\
    \hline
    freeQueue  & キューのデータを解放する関数                     \\
    \hline
  \end{tabular}
\end{table}
それぞれの関数について説明します。

\subsubsection{enqueue関数}
enqueue関数は表\ref{table:listenqueue}のように定義され、
リスト\ref{listlistenqueue}のように実装されています。

\begin{table}[htbp]
  \centering
  \caption{リストキューのenqueue関数}
  \label{table:listenqueue}
  \begin{tabular}{|c|c|}
    \hline
    関数名 & enqueue              \\
    \hline
    型   & int                  \\
    \hline
    引数  & int data:キューに格納するデータ \\
    \hline
    戻り値 & 実行の成否を返す             \\
    \hline
    概要  & キューにデータを積み上げる関数      \\
    \hline
  \end{tabular}
\end{table}

\begin{lstlisting}[caption=リストキューのenqueue関数の実装,label=listlistenqueue]
  int enqueue(int data) {
      int r_value = -100;
      struct queue *new_queue;
      if (data <= 0) {
          r_value = -102;
      } else if ((new_queue = (struct queue *)malloc(sizeof(struct queue))) ==
              NULL) {
          r_value = -101;
      } else {
          new_queue->value = data;
          new_queue->addr = NULL;
          if (bottom_queue == NULL) {
              bottom_queue = new_queue;
          } else {
              top_queue->addr = new_queue;
          }
          top_queue = new_queue;
      }
      return r_value;
  }
\end{lstlisting}

enqueue関数は、次のような処理が実装されています。
\begin{enumerate}
  \item データが自然数か確認し、自然数でなかったらエラーコードを返す。
  \item メモリを確保できなかった場合はエラーコードを返す。
  \item メモリを確保できた場合は、データを格納し、次のキューのアドレスをNULLにする。
  \item 最古のキューのアドレスがNULLの場合は、新しいキューのアドレスを最古のキューのアドレスにする。
        そうでない場合は、最新のキューのアドレスのアドレスに新しいキューのアドレスを格納する。
  \item 最新のキューのアドレスを新しいキューのアドレスにする。
  \item 成功を示す返り値を返す。
\end{enumerate}

\subsubsection{dequeue関数}
dequeue関数は表\ref{table:listdequeue}のように定義され、
リスト\ref{listlistdequeue}のように実装されています。

\begin{table}[htbp]
  \centering
  \caption{リストキューのdequeue関数}
  \label{table:listdequeue}
  \begin{tabular}{|c|c|}
    \hline
    関数名 & dequeue             \\
    \hline
    型   & int                 \\
    \hline
    引数  & なし                  \\
    \hline
    戻り値 & 取り出したデータ、もしくはエラーコード \\
    \hline
    概要  & キューからデータを取り出す関数     \\
    \hline
  \end{tabular}
\end{table}

\begin{lstlisting}[caption=リストキューのdequeue関数の実装,label=listlistdequeue]
  int dequeue() {
      int r_value = -200;
      struct queue *new_bottom;
      if (top_queue == bottom_queue &&
          bottom_queue != NULL) {  // キューにデータが一つしかない場合
          r_value = bottom_queue->value;
          free(bottom_queue);
          bottom_queue = NULL;
          top_queue = NULL;
      } else if (bottom_queue != NULL &&
               top_queue != NULL) {  // キューにデータが複数ある場合
          r_value = bottom_queue->value;
          new_bottom = bottom_queue->addr;
          free(bottom_queue);
          bottom_queue = new_bottom;
      } else if (bottom_queue == NULL) {  // キューにデータがない場合
          r_value = -201;
      } else {  // その他のエラー
          r_value = -202;
      }
      return r_value;
  }
\end{lstlisting}

dequeue関数は、次のような処理が実装されています。
\begin{enumerate}
  \item キューにデータが一つしかない場合、最新のキューの値を
        取得し、最古のキューのアドレスを解放し、NULLにする。
  \item キューにデータが複数ある場合、最新のキューの値を
        取得し、最新のキューのアドレスを最古のキューの
        アドレスにする。
  \item キューにデータがない場合はエラーコードを返す。
  \item その他のエラーの場合はエラーコードを返す。
\end{enumerate}

\subsubsection{showQueue関数}
showQueue関数は表\ref{table:listshowQueue}のように定義され、
リスト\ref{listlistshowQueue}のように実装されています。

\begin{table}[htbp]
  \centering
  \caption{リストキューのshowQueue関数}
  \label{table:listshowQueue}
  \begin{tabular}{|c|c|}
    \hline
    関数名 & showQueue      \\
    \hline
    型   & int            \\
    \hline
    引数  & なし             \\
    \hline
    戻り値 & 成功したら0を返す      \\
    \hline
    概要  & キューのデータを表示する関数 \\
    \hline
  \end{tabular}
\end{table}

\begin{lstlisting}[caption=リストキューのshowQueue関数の実装,label=listlistshowQueue]
  int showQueue() {
      struct queue *print_queue;
      print_queue = bottom_queue;
      printf("|");
      while (print_queue != NULL) {
          printf("%d|", print_queue->value);
          print_queue = print_queue->addr;
      }
      return 0;
  }
\end{lstlisting}

showQueue関数は、次のような処理が実装されています。
\begin{enumerate}
  \item データとデータの間に区切り文字「|」を表示する。
  \item キューのデータを表示する。
  \item 成功を示す返り値を返す。
\end{enumerate}

\subsubsection{showResult関数}
showResult関数は表\ref{table:listshowResult}のように定義され、
リスト\ref{listlistshowResult}のように実装されています。

\begin{table}[htbp]
  \centering
  \caption{リストキューのshowResult関数}
  \label{table:listshowResult}
  \begin{tabular}{|c|c|}
    \hline
    関数名 & showResult                         \\
    \hline
    型   & void                               \\
    \hline
    引数  & int result:エラーコード                  \\
    \hline
    戻り値 & なし                                 \\
    \hline
    概要  & 変数resultの値に応じて、対応するエラーメッセージを表示する関数 \\
    \hline
  \end{tabular}
\end{table}

\begin{lstlisting}[caption=リストキューのshowResult関数の実装,label=listlistshowResult]
  void showResult(int result) {
      switch (result) {
          case 0:
              printf("Print success\n");
              break;
          case -100:
              printf("Enqueue success\n");
              break;
          case -101:
              printf("Enqueue error: no remaining area(ErrorCode-101)\n");
              break;
          case -102:
              printf(
                "Enqueue error:Enqueue error:Not natural number was "
                "entered:(ErrorCode-102)\n");
              break;
          case -200:
              printf("Dequeue error:Other error(ErrorCode-200)\n");
              break;
          case -201:
              printf("Dequeue error:Nothing data(ErrorCode-201)\n");
              break;
          case -202:
              printf("Dequeue error:Dequeue error(ErrorCode-202)\n");
              break;
          default:
              printf("Dequeue success\n");
              break;
      }
  }
\end{lstlisting}

showResult関数は、次のような処理が実装されています。
\begin{enumerate}
  \item resultの値を、次の項目に分けてエラーメッセージを表示する。
        \begin{itemize}
          \item 0:表示成功
          \item -100:エンキュー成功
          \item -101:エンキューエラー:残り領域なし
          \item -102:エンキューエラー:自然数が入力されていない
          \item -200:デキューエラー:その他のエラー
          \item -201:デキューエラー:データがない
          \item -202:デキューエラー:デキューエラー
          \item その他:デキュー成功
        \end{itemize}
\end{enumerate}

\subsubsection{freeQueue関数}
freeQueue関数は表\ref{table:freeQueue}のように定義され、
リスト\ref{listfreeQueue}のように実装されています。

\begin{table}[htbp]
  \centering
  \caption{リストキューのfreeQueue関数}
  \label{table:freeQueue}
  \begin{tabular}{|c|c|}
    \hline
    関数名 & freeQueue      \\
    \hline
    型   & void           \\
    \hline
    引数  & なし             \\
    \hline
    戻り値 & なし             \\
    \hline
    概要  & キューのデータを解放する関数 \\
    \hline
  \end{tabular}
\end{table}

\begin{lstlisting}[caption=リストキューのfreeQueue関数の実装,label=listfreeQueue]
  void freeQueue() {
      struct queue *this_Queue;
      while (bottom_queue != NULL) {
          this_Queue = bottom_queue->addr;
          free(bottom_queue);
          bottom_queue = this_Queue;
      }
  }
\end{lstlisting}

freeQueue関数は、次のような処理が実装されています。
\begin{enumerate}
  \item 古いキューから順にメモリを解放する。
\end{enumerate}

\section{プログラム}
この章では、再構築するキュー、リングキュー、リストキューの
メイン関数を示し、処理について説明します。なおそれぞれの
プログラム全文は長いので、レポートの最後の付録に示します。

\subsection{再構築するキュー}
再構築するキューのメイン関数をリスト\ref{mainqueue}に示します。

\begin{lstlisting}[caption=再構築するキューのメイン関数,label=mainqueue]
  int main() {
      int i;
      int dequeueData, dequeueResult;
      // キューの初期化
      printf("Queue initialize");
      showResult(initQueue());
      // キューにデータを積み上げる
      for (i = 1; i <= 5; i++) {
          showQueue();
          printf("<--%d", i);
          showResult(enqueue(i));
      }
      // キューのデータを表示する
      // キューからデータを取り出す
      for (i = 0; i < 5; i++) {
          dequeueData = queue[0];
          dequeueResult = dequeue();
          showQueue();
          printf("-->%d", dequeueData);
          showResult(dequeueResult);
      }
      // キューのデータを表示する
      for(i = 2; i >= 0;i--){
          showQueue();
          printf("<--%d", i);
          showResult(enqueue(i));
      }
      return 0;
  }
\end{lstlisting}

メイン関数では次の処理が行われています。
\begin{enumerate}
  \item キューの初期化を行う。
  \item キューにデータを1から5まで積み上げる。
  \item キューからデータを5つ取り出す。
  \item キューにデータを2から0まで積み上げる。
\end{enumerate}

\subsection{リングキュー}
リングキューのメイン関数をリスト\ref{mainringqueue}に示します。

\begin{lstlisting}[caption=リングキューのメイン関数,label=mainringqueue]
  int main() {
      Queue queue;
      int dequeueData, dequeueResult;
      int i;
      // キューの初期化
      printf("Queue initialize");
      showResult(initQueue(&queue));
      // データがいっぱいになった時の挙動を確認
      for (i = 1; i <= 6; i++) {
          showQueue(&queue);
          printf("<--%d", i);
          showResult(enqueue(&queue, i));
      }

      // 空の時にデータを取り出す時の挙動を確認
      for (i = 0; i < 6; i++) {
          dequeueData = queue.data[dequeueIndex(queue)];
          dequeueResult = dequeue(&queue);
          showQueue(&queue);
          printf("-->%d", dequeueData);
          showResult(dequeueResult);
      }
      // 自然数以外の値を入力した時の挙動を確認
      showQueue(&queue);
      printf("<--%d", 0);
      showResult(enqueue(&queue, 0));
      // キューがリングバッファであることを確認
      showQueue(&queue);
      printf("<--%d", 1);
      showResult(enqueue(&queue, 1));
      showQueue(&queue);
      printf("<--%d", 2);
      showResult(enqueue(&queue, 2));
      for(i = 3;i < 10;i++){
          showQueue(&queue);
          printf("<--%d", i);
          showResult(enqueue(&queue, i));
          dequeueData = queue.data[dequeueIndex(queue)];
          dequeueResult = dequeue(&queue);
          showQueue(&queue);
          printf("-->%d", dequeueData);
          showResult(dequeueResult);
      }
      return 0;
  }
\end{lstlisting}

メイン関数では次の処理が行われています。
\begin{enumerate}
  \item キューの初期化を行う。
  \item キューにデータを1から6まで積み上げる。
  \item キューからデータを6つ取り出す。
  \item キューに0を積み上げる。
  \item キューに1から2まで積み上げる。
  \item キューにデータを積み上げ、取り出すのを3から9まで繰り返す。
\end{enumerate}

\subsection{リストキュー}
リストキューのメイン関数をリスト\ref{mainlistqueue}に示します。

\begin{lstlisting}[caption=リストキューのメイン関数,label=mainlistqueue]
  int main(void) {
      int i;
      int dequeueData, dequeueResult;
      // データをキューに積み上げる
      for (i = 1; i < 10; i++) {
          showQueue();
          printf("<--%d:", i);
          showResult(enqueue(i));
      }
      // データをキューから取り出す
      for (i = 0; i < 10; i++) {
          dequeueResult = dequeue();
          if (dequeueResult > 0) {
              showQueue();
              printf("-->%d:", dequeueResult);
              showResult(dequeueResult);
          } else {
              showResult(dequeueResult);
          }
      }
      for (i = 2; i >= 0; i--) {
          showQueue();
          printf("<--%d:", i);
          showResult(enqueue(i));
      }
      return 0;
  }
\end{lstlisting}

メイン関数では次の処理が行われています。
\begin{enumerate}
  \item キューにデータを1から9まで積み上げる。
  \item キューからデータを10個取り出す。
  \item キューにデータを2から0まで積み上げる。
\end{enumerate}

\section{実行結果}
この章では、再構築するキュー、リングキュー、リストキューの
実行結果を示し、課題の内容に沿っているか確認します。

\subsection{再構築するキュー}
再構築するキューの実行結果をリスト\ref{resultqueue}に示します。

\begin{lstlisting}[caption=再構築するキューの実行結果,label=resultqueue]
  Queue initialize:success
  |0|0|0|0|<--1:Enqueue success
  |1|0|0|0|<--2:Enqueue success
  |1|2|0|0|<--3:Enqueue success
  |1|2|3|0|<--4:Enqueue success
  |1|2|3|4|<--5:Enqueue error:No space left(ErrorCode-101)
  |2|3|4|0|-->1:Dequeue success
  |3|4|0|0|-->2:Dequeue success
  |4|0|0|0|-->3:Dequeue success
  |0|0|0|0|-->4:Dequeue success
  |0|0|0|0|-->0:Dequeue error:Nothing data(ErrorCode-201)
  |0|0|0|0|<--2:Enqueue success
  |2|0|0|0|<--1:Enqueue success
  |2|1|0|0|<--0:Enqueue error:Not natural number was entered:(ErrorCode-102)
\end{lstlisting}

再構築するキューでは以下の事項を確認できるように実装されています。
\begin{itemize}
  \item データがいっぱいになった時にエラー処理ができているか。
  \item 空の時にデータを取り出す時のエラー処理ができているか。
  \item 自然数以外の値を入力した時のエラー処理ができているか。
  \item 再構築が行われているか。
\end{itemize}

それぞれについて説明します。

\subsubsection{データがいっぱいになった時にエラー処理ができているか}
リスト\ref{resultqueue}の2から6行目を見ると、
キューの最大容量は4であるため、4つのデータを積み上げた後に
5つ目のデータを積み上げるとエラーが表示されてます。
よって、データがいっぱいになった時にエラー処理ができていることが
確認できます。

\subsubsection{空の時にデータを取り出す時のエラー処理ができているか}
リスト\ref{resultqueue}の11行目を見ると、
データがない状態でデータを取り出そうとするとエラーが表示されて
いることがわかります。
よって、空の時にデータを取り出す時のエラー処理ができていることが
確認できます。

\subsubsection{自然数以外の値を入力した時のエラー処理ができているか}
リスト\ref{resultqueue}の14行目を見ると、0を
エンキューしようとした時にエラーが表示されていることがわかります。
よって、自然数以外の値を入力した時のエラー処理ができていることが
確認できます。

\subsubsection{再構築が行われているか}
リスト\ref{resultqueue}をみると、デキューした際に、すべての
データが左揃えに直されているので、再構築が行われていることが
確認できます。

\subsection{リングキュー}
リングキューの実行結果をリスト\ref{resultringqueue}に示します。

\begin{lstlisting}[caption=リングキューの実行結果,label=resultringqueue]
Queue initialize:Init success
|0|0|0|0|0|<--1:Enqueue success
|1|0|0|0|0|<--2:Enqueue success
|1|2|0|0|0|<--3:Enqueue success
|1|2|3|0|0|<--4:Enqueue success
|1|2|3|4|0|<--5:Enqueue success
|1|2|3|4|5|<--6:Enqueue error:No space left(ErrorCode-101)
|0|2|3|4|5|-->1:Dequeue success
|0|0|3|4|5|-->2:Dequeue success
|0|0|0|4|5|-->3:Dequeue success
|0|0|0|0|5|-->4:Dequeue success
|0|0|0|0|0|-->5:Dequeue success
|0|0|0|0|0|-->0:Dequeue error:Nothing data(ErrorCode-201)
|0|0|0|0|0|<--0:Enqueue error:Not natural number was entered:(ErrorCode-102)
|0|0|0|0|0|<--1:Enqueue success
|1|0|0|0|0|<--2:Enqueue success
|1|2|0|0|0|<--3:Enqueue success
|0|2|3|0|0|-->1:Dequeue success
|0|2|3|0|0|<--4:Enqueue success
|0|0|3|4|0|-->2:Dequeue success
|0|0|3|4|0|<--5:Enqueue success
|0|0|0|4|5|-->3:Dequeue success
|0|0|0|4|5|<--6:Enqueue success
|6|0|0|0|5|-->4:Dequeue success
|6|0|0|0|5|<--7:Enqueue success
|6|7|0|0|0|-->5:Dequeue success
|6|7|0|0|0|<--8:Enqueue success
|0|7|8|0|0|-->6:Dequeue success
|0|7|8|0|0|<--9:Enqueue success
|0|0|8|9|0|-->7:Dequeue success
\end{lstlisting}

リングキューでは以下の事項を確認できるように実装されています。
\begin{itemize}
  \item データがいっぱいになった時にエラー処理ができているか。
  \item 空の時にデータを取り出す時のエラー処理ができているか。
  \item 自然数以外の値を入力した時のエラー処理ができているか。
  \item キューがリングバッファであることを確認できるか。
\end{itemize}

それぞれについて説明します。

\subsubsection{データがいっぱいになった時にエラー処理ができているか}
リスト\ref{resultringqueue}の6行目を見ると、
キューの最大容量は5であるため、5つのデータを積み上げた後に
6つ目のデータを積み上げるとエラーが表示されてます。
よって、データがいっぱいになった時にエラー処理ができていることが
確認できます。

\subsubsection{空の時にデータを取り出す時のエラー処理ができているか}
リスト\ref{resultringqueue}の13行目を見ると、
データがない状態でデータを取り出そうとするとエラーが表示されて
いることがわかります。
よって、空の時にデータを取り出す時のエラー処理ができていることが
確認できます。

\subsubsection{自然数以外の値を入力した時のエラー処理ができているか}
リスト\ref{resultringqueue}の14行目を見ると、0を
エンキューしようとした時にエラーが表示されていることがわかります。
よって、自然数以外の値を入力した時のエラー処理ができていることが
確認できます。

\subsubsection{キューがリングバッファであることを確認できるか}
リスト\ref{resultringqueue}の17行目から最後までを見ると、
データを積み上げ、取り出すのを繰り返すと、データが端にそろわず
移動していくため、リングバッファであることが確認できます。

\subsection{リストキュー}
リストキューの実行結果をリスト\ref{resultlistqueue}に示します。

\begin{lstlisting}[caption=リストキューの実行結果,label=resultlistqueue]
|<--1:Enqueue success
|1|<--2:Enqueue success
|1|2|<--3:Enqueue success
|1|2|3|<--4:Enqueue success
|1|2|3|4|<--5:Enqueue success
|1|2|3|4|5|<--6:Enqueue success
|1|2|3|4|5|6|<--7:Enqueue success
|1|2|3|4|5|6|7|<--8:Enqueue success
|1|2|3|4|5|6|7|8|<--9:Enqueue success
|2|3|4|5|6|7|8|9|-->1:Dequeue success
|3|4|5|6|7|8|9|-->2:Dequeue success
|4|5|6|7|8|9|-->3:Dequeue success
|5|6|7|8|9|-->4:Dequeue success
|6|7|8|9|-->5:Dequeue success
|7|8|9|-->6:Dequeue success
|8|9|-->7:Dequeue success
|9|-->8:Dequeue success
|-->9:Dequeue success
Dequeue error:Nothing data(ErrorCode-201)
|<--2:Enqueue success
|2|<--1:Enqueue success
|2|1|<--0:Enqueue error:Enqueue error:Not natural number was entered:(ErrorCode-102)
\end{lstlisting}

リストキューでは以下の事項を確認できるように実装されています。
\begin{itemize}
  \item 空の時にデータを取り出す時のエラー処理ができているか。
  \item 自然数以外の値を入力した時のエラー処理ができているか。
  \item リストになっていることが確認できるか。
\end{itemize}

それぞれについて説明します。

\subsubsection{空の時にデータを取り出す時のエラー処理ができているか}
リスト\ref{resultlistqueue}の18行目を見ると、
データがない状態でデータを取り出そうとするとエラーが表示されて
いることがわかります。
よって、空の時にデータを取り出す時のエラー処理ができていることが
確認できます。

\subsubsection{自然数以外の値を入力した時のエラー処理ができているか}
リスト\ref{resultlistqueue}の22行目を見ると、0を
エンキューしようとした時にエラーが表示されていることがわかります。
よって、自然数以外の値を入力した時のエラー処理ができていることが
確認できます。

\subsubsection{リストになっていることが確認できるか}
リスト\ref{resultlistqueue}を見ると、データを積み上げた
分だけのキューのみ表示されているので、リストになっていることが
確認できます。


\section{考察}
この章では、再構築するキュー、リングキュー、リストキューの
実行結果をもとに、それぞれのキューの特徴について考察します。

\subsection{再構築するキュー}
再構築するキューは、固定長の配列を使用して実装されています。
デキュー、エンキューのたびにデータを再構築するため、
データが多い場合は処理が遅くなります。

\subsection{リングキュー}
リングキューは、固定長の配列を使用して実装されています。
再構築するキューとちがい、容量を効率よく使え、無駄がありません。

\subsection{リストキュー}
リストキューは、連結リストを使用して実装されています。
連結リストの長さに制限はありませんので、理論上は無限のデータを追加できます。
そのため、データの個数を制限したい場合などは向きません。

\section{付録}
この章では、再構築するキュー、リングキュー、リストキューのプログラム
全文を示します。

\subsection{再構築するキュー}
再構築するキューのプログラム全文をリスト\ref{queue}に示します。

\begin{lstlisting}[caption=再構築するキューのプログラム,label=queue]
  #include <stdio.h>

  #define QUEUE_SIZE 4
  
  int queue[QUEUE_SIZE]; // キュー用の配列
  int quantity = 0; // キューに格納されているデータの個数
  
  /// @brief キューにデータを積み上げる
  /// @param data キューに格納するデータ
  /// @return
  /// 実行の成否を返す。成功:-100,残領域がない:-101,0以下の値が入力された:-102
  int enqueue(int data) {
      // 残り領域があるか確認する
      if (quantity >= QUEUE_SIZE) {
          // 残り領域がない場合はエラーコードを返す
          return -101;
      }
      // データが自然数か確認する
      if (data <= 0) {
          // 自然数でない場合はエラーコードを返す
          return -102;
      }
      // 配列のキューにデータを保存する
      queue[quantity] = data;
      // キューのポインタをインクリメントする
      quantity++;
      // 返り値を返す
      return -100;
  }
  
  /// @brief キューからデータを取り出す
  /// @return
  /// 取り出したデータ,もしくはエラーコード(取り出せるデータがなかった時:-201)
  int dequeue(void) {
      // データが存在するかどうか確認する.
      if (quantity == 0) {
          // データが存在しない場合はエラーコードを返す.
          return -201;
      }
      // キューからデータをとりだす.
      int data = queue[0];
      // データの個数カウントを減らす
      quantity--;
      // 取り出されたデータ部分を埋めるように再構築する
      for (int i = 0; i < quantity; i++) {
          queue[i] = queue[i + 1];
      }
      queue[quantity] = 0;
      // データを返す
      return data;
  }
  
  /// @brief キュー用の配列を0で初期化する。
  /// @return 成功したら0を返す。
  int initQueue() {
      // キューのデータを入れる配列をすべて 0 に初期化する.
      for (int i = 0; i < QUEUE_SIZE; i++) {
          queue[i] = 0;
      }
      // 格納データ個数を 0 に初期化する.
      quantity = 0;
      return 0;
  }
  
  int showQueue() {
      // 配列全体のデータを順に表示する.
      printf("|");
      for (int i = 0; i < QUEUE_SIZE; i++) {
          printf("%d", queue[i]);
          // データとデータの間に区切り文字「|」を表示する.
          printf("|");
      }
      return 0;
  }
  
  void showResult(int result) {
      // result の値に応じて,対応するエラーメッセージを表示する.
      switch (result) {
          case -100:
              printf(":Enqueue success\n");
              break;
          case -101:
              printf(":Enqueue error:No space left(ErrorCode-101)\n");
              break;
          case -102:
              printf(":Enqueue error:Not natural number was entered:(ErrorCode-102)\n");
              break;
          case -201:
              printf(":Dequeue error:Nothing data(ErrorCode-201)\n");
              break;
          case 0:
              printf(":success\n");
              break;
          default:
              printf(":Dequeue success\n");
              break;
      }
  }
  
  int main() {
      int i;
      int dequeueData, dequeueResult;
      // キューの初期化
      printf("Queue initialize");
      showResult(initQueue());
      // キューにデータを積み上げる
      for (i = 1; i <= 5; i++) {
          showQueue();
          printf("<--%d", i);
          showResult(enqueue(i));
      }
      // キューのデータを表示する
      // キューからデータを取り出す
      for (i = 0; i < 5; i++) {
          dequeueData = queue[0];
          dequeueResult = dequeue();
          showQueue();
          printf("-->%d", dequeueData);
          showResult(dequeueResult);
      }
      // キューのデータを表示する
      for(i = 2; i >= 0;i--){
          showQueue();
          printf("<--%d", i);
          showResult(enqueue(i));
      }
      return 0;
  }
\end{lstlisting}

\subsection{リングキュー}
リングキューのプログラム全文をリスト\ref{ringqueue}に示します。

\begin{lstlisting}[caption=リングキューのプログラム,label=ringqueue]
  #include <stdio.h>

#define QUEUE_SIZE 5

typedef struct {
    int data[QUEUE_SIZE];  // データが入る配列
    int wp;                // 次にデータを入れる場所の配列番号
    int quantity;          // データの個数
} Queue;

/// @brief キューからデータを取り出す際のインデックスを返す
/// @param obj データを取り出すキューの構造体
/// @return キューからデータを取り出す際のインデックス
int dequeueIndex(Queue obj) {
    if (obj.wp - obj.quantity < 0) {
        return QUEUE_SIZE + obj.wp - obj.quantity;
    } else {
        return obj.wp - obj.quantity;
    }
}

/// @brief キューにデータを積み上げる
/// @param obj データを追加するキューの構造体のアドレス
/// @param data キューに格納するデータ
/// @return
/// 実行の成否を返す。成功:-100,残領域がない:-101,0以下の値が入力された:-102
int enqueue(Queue *obj, int data) {
    // 残り領域があるか確認する
    if (obj->quantity >= QUEUE_SIZE) {
        // 残り領域がない場合はエラーコードを返す
        return -101;
    }
    // データが自然数か確認する
    if (data <= 0) {
        // 自然数でない場合はエラーコードを返す
        return -102;
    }
    // 配列のキューにデータを保存する
    obj->data[obj->wp] = data;
    // キューのポインタをインクリメントする
    obj->quantity++;
    if (obj->wp == QUEUE_SIZE - 1) {
        obj->wp = 0;
    } else {
        obj->wp++;
    }
    // 返り値を返す
    return -100;
}

/// @brief キューからデータを取り出す
/// @param obj データを取り出すキューの構造体のアドレス
/// @return
/// 取り出したデータ,もしくはエラーコード(取り出せるデータがなかった時:-201)
int dequeue(Queue *obj) {
    int index;
    index = dequeueIndex(*obj);
    // データが存在するかどうか確認する.
    if (obj->quantity == 0) {
        // データが存在しない場合はエラーコードを返す.
        return -201;
    }
    // キューからデータをとりだす.
    int data = obj->data[index];
    // 取り出されたデータ部分を埋めるように再構築する
    obj->data[index] = 0;
    // データの個数のカウントを減らす.
    obj->quantity--;
    // データを返す.
    return data;
}

/// @brief キュー用の配列を0で初期化する。
/// @param obj キューの構造体のアドレス
/// @return 成功したら-400を返す。
int initQueue(Queue *obj) {
    int i;
    // キューのデータを入れる配列をすべて 0 に初期化する.
    for (i = 0; i < QUEUE_SIZE; i++) {
        obj->data[i] = 0;
    }
    // キューのデータ格納個数を 0 に初期化する
    obj->quantity = 0;
    // キューの wp ポインタを 0 に初期化する.
    obj->wp = 0;
    return -400;
}

/// @brief キューのデータを表示する
/// @param obj キューの構造体のアドレス
/// @return 成功したら-500を返す。
int showQueue(Queue *obj) {
    int i;
    // 配列全体のデータを順に表示する.
    // データとデータの間に区切り文字「|」を表示する.
    printf("|");
    for (i = 0; i < QUEUE_SIZE; i++) {
        printf("%d", obj->data[i]);
        printf("|");
    }
    return -500;
}

/// @brief 動作に対しての結果を表示する
/// @param result 結果の値
void showResult(int result) {
    switch (result) {
        case -100:
            printf(":Enqueue success\n");
            break;
        case -101:
            printf(":Enqueue error:No space left(ErrorCode-101)\n");
            break;
        case -102:
            printf(
                ":Enqueue error:Not natural number was "
                "entered:(ErrorCode-102)\n");
            break;
        case -201:
            printf(":Dequeue error:Nothing data(ErrorCode-201)\n");
            break;
        case -400:
            printf(":Init success\n");
            break;
        case -500:
            printf(":Print success\n");
            break;
        default:
            printf(":Dequeue success\n");
            break;
    }
    // result の値に応じて,対応するエラーメッセージを表示する.
}

int main() {
    Queue queue;
    int dequeueData, dequeueResult;
    int i;
    // キューの初期化
    printf("Queue initialize");
    showResult(initQueue(&queue));
    // データがいっぱいになった時の挙動を確認
    for (i = 1; i <= 6; i++) {
        showQueue(&queue);
        printf("<--%d", i);
        showResult(enqueue(&queue, i));
    }

    // 空の時にデータを取り出す時の挙動を確認
    for (i = 0; i < 6; i++) {
        dequeueData = queue.data[dequeueIndex(queue)];
        dequeueResult = dequeue(&queue);
        showQueue(&queue);
        printf("-->%d", dequeueData);
        showResult(dequeueResult);
    }
    // 自然数以外の値を入力した時の挙動を確認
    showQueue(&queue);
    printf("<--%d", 0);
    showResult(enqueue(&queue, 0));
    // キューがリングバッファであることを確認
    showQueue(&queue);
    printf("<--%d", 1);
    showResult(enqueue(&queue, 1));
    showQueue(&queue);
    printf("<--%d", 2);
    showResult(enqueue(&queue, 2));
    for (i = 3; i < 10; i++) {
        showQueue(&queue);
        printf("<--%d", i);
        showResult(enqueue(&queue, i));
        dequeueData = queue.data[dequeueIndex(queue)];
        dequeueResult = dequeue(&queue);
        showQueue(&queue);
        printf("-->%d", dequeueData);
        showResult(dequeueResult);
    }
    return 0;
}
\end{lstlisting}

\subsection{リストキュー}
リストキューのプログラム全文をリスト\ref{listqueue}に示します。

\begin{lstlisting}[caption=リストキューのプログラム,label=listqueue]
  #include <stdio.h>
  #include <stdlib.h>
  
  struct queue {
      int value;
      struct queue *addr;
  };
  
  struct queue *bottom_queue =
      NULL;  // 最古のキューのアドレスを記憶しておくポインタ
  struct queue *top_queue = NULL;  // 最新のキューのアドレスを記憶しておくポインタ
  
  /// @brief キューにデータを積み上げる
  /// @param data キューに格納するデータ
  /// @return
  /// 実行の成否を返す。成功:-100,残領域がない:-101,0以下の値が入力された:-102
  int enqueue(int data) {
      int r_value = -100;
      struct queue *new_queue;
      if (data <= 0) {
          r_value = -102;
      } else if ((new_queue = (struct queue *)malloc(sizeof(struct queue))) ==
                 NULL) {
          r_value = -101;
      } else {
          new_queue->value = data;
          new_queue->addr = NULL;
          if (bottom_queue == NULL) {
              bottom_queue = new_queue;
          } else {
              top_queue->addr = new_queue;
          }
          top_queue = new_queue;
      }
      return r_value;
  }
  
  /// @brief キューからデータを取り出す
  /// @return
  /// 取り出したデータ,もしくはエラーコード(取り出せるデータがなかった時:-201,
  /// キュ取り出しエラー:-202,その他のエラー:-200)
  int dequeue() {
      int r_value = -200;
      struct queue *new_bottom;
      if (top_queue == bottom_queue &&
          bottom_queue != NULL) {  // キューにデータが一つしかない場合
          r_value = bottom_queue->value;
          free(bottom_queue);
          bottom_queue = NULL;
          top_queue = NULL;
      } else if (bottom_queue != NULL &&
                 top_queue != NULL) {  // キューにデータが複数ある場合
          r_value = bottom_queue->value;
          new_bottom = bottom_queue->addr;
          free(bottom_queue);
          bottom_queue = new_bottom;
      } else if (bottom_queue == NULL) {  // キューにデータがない場合
          r_value = -201;
      } else {  // その他のエラー
          r_value = -202;
      }
      return r_value;
  }
  
  /// @brief キュー用の配列のデータを表示する。
  /// @return 成功したら0を返す。
  int showQueue() {
      struct queue *print_queue;
      print_queue = bottom_queue;
      printf("|");
      while (print_queue != NULL) {
          printf("%d|", print_queue->value);
          print_queue = print_queue->addr;
      }
      return 0;
  }
  
  /// @brief エラーメッセージを表示する
  /// @param result エラーコード
  void showResult(int result) {
      switch (result) {
          case 0:
              printf("Print success\n");
              break;
          case -100:
              printf("Enqueue success\n");
              break;
          case -101:
              printf("Enqueue error: no remaining area(ErrorCode-101)\n");
              break;
          case -102:
              printf(
                  "Enqueue error:Enqueue error:Not natural number was "
                  "entered:(ErrorCode-102)\n");
              break;
          case -200:
              printf("Dequeue error:Other error(ErrorCode-200)\n");
              break;
          case -201:
              printf("Dequeue error:Nothing data(ErrorCode-201)\n");
              break;
          case -202:
              printf("Dequeue error:Dequeue error(ErrorCode-202)\n");
              break;
          default:
              printf("Dequeue success\n");
              break;
      }
  }
  
  /// @brief キューのデータを解放する
  void freeQueue() {
      struct queue *this_Queue;
      while (bottom_queue != NULL) {
          this_Queue = bottom_queue->addr;
          free(bottom_queue);
          bottom_queue = this_Queue;
      }
  }
  
  int main(void) {
      int i;
      int dequeueData, dequeueResult;
      // データをキューに積み上げる
      for (i = 1; i < 10; i++) {
          showQueue();
          printf("<--%d:", i);
          showResult(enqueue(i));
      }
      // データをキューから取り出す
      for (i = 0; i < 10; i++) {
          dequeueResult = dequeue();
          if (dequeueResult > 0) {
              showQueue();
              printf("-->%d:", dequeueResult);
              showResult(dequeueResult);
          } else {
              showResult(dequeueResult);
          }
      }
      for (i = 2; i >= 0; i--) {
          showQueue();
          printf("<--%d:", i);
          showResult(enqueue(i));
      }
      return 0;
  }
\end{lstlisting}

\end{document}