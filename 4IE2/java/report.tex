\documentclass[a4paper,11pt,dvipdfmx]{jsarticle}


% 数式
\usepackage{amsmath,amsfonts}
\usepackage{bm}
\usepackage{physics}
\usepackage{mathtools}
% 画像
\usepackage[dvipdfmx]{graphicx}
\usepackage{circuitikz}
\usepackage{amsmath,amssymb}
\usepackage{siunitx}
\usepackage{float}
\usepackage{tikz}
\usepackage{askmaps}
\usepackage{multirow}
\usepackage{bigstrut}
\usepackage{rotating}
\usepackage{listings}
\usepackage{subcaption}
% 表
\usepackage{makecell}
% その他
\usepackage{url}
\usepackage{ascmac}
\usepackage{cases}
\usepackage{here}
\usepackage{upgreek}
\usepackage{tocloft}  % tocloftパッケージを使う
\usepackage{titlesec} % titlesecパッケージを使う(セクションタイトルのカスタマイズ)

% 画像挿入コマンド
\newcommand{\Figure}[4]{
\begin{figure}[H]
\centering
\includegraphics[width=#1\linewidth]{./images/#2}
\caption{#3}
\label{fig:#4}
\end{figure}
}

\lstset{
basicstyle={\ttfamily},
identifierstyle={\small},
commentstyle={\smallitshape},
keywordstyle={\small\bfseries},
ndkeywordstyle={\small},
stringstyle={\small\ttfamily},
frame={tb},
breaklines=true,
columns=[l]{fullflexible},
numbers=left,
xrightmargin=0zw,
xleftmargin=3zw,
numberstyle={\scriptsize},
stepnumber=1,
numbersep=1zw,
lineskip=-0.5ex
}
\renewcommand{\lstlistingname}{リスト}

\begin{document}

\title{starategy}
\author{学籍番号:22120 \\ 組番号:222 \\名前:塚田 勇人}
\date{\today}
\maketitle

\newpage

\section{プログラミングリスト}
playerクラスをリスト\ref{player}に示す。
\begin{lstlisting}[caption=playerクラス,label=player]
  class player {
    private String name;// プレイヤー名を記憶する
    private Strategy strategy;// 戦略のインスタンス
    private int wincount;// 勝った回数を記憶する
    private int losecount;// 負けた回数を記憶する
    // プレイヤーをインスタンス化するときに,名前と戦略を決める

    player(String name, Strategy strategy) {
        this.name = name;
        this.strategy = strategy;
    }

    // i 回戦の対戦カードを出す
    public int nextNumber(int i) {
        return strategy.nextNumber(i);
    }

    // 敵が出したカードの数を渡される
    public void learning(int enemy) {
        strategy.learning(enemy);
    }

    // 勝ったら勝ち回数をインクリメントする
    public void win() {
        wincount++;
    }

    // 負けたら負け回数をインクリメントする
    public void lose() {
        losecount++;
    }

    // 呼ばれたときの勝ち数と負け数の文字列を返す
    public String toString() {
        return (wincount + "勝" + losecount + "敗");
    }

    // プレイヤーの名前を返す
    public String getName() {
        return name;
    }
}
\end{lstlisting}

strategyクラスをリスト\ref{strategy}に示す。

\begin{lstlisting}[caption=strategyクラス,label=strategy]
  interface Strategy {
    public abstract Integer nextNumber(int i);

    public abstract void learning(int enemyNumber);
}
\end{lstlisting}

strategyFollowクラスをリスト\ref{strategyFollow}に示す。

\begin{lstlisting}[caption=strategyFollowクラス,label=strategyFollow]
  import java.util.ArrayList;
  import java.util.List;
  
  class strategyFollow implements Strategy {
      // 直前に相手が出した番号を記憶しておくフィールド
      private int preValue = 1;
      // 手持ちの札を覚えておくリスト
      private List<Integer> cards = new ArrayList<Integer>();
  
      strategyFollow() {
          cards.add(1);
          cards.add(2);
          cards.add(3);
          cards.add(4);
          cards.add(5);
      }
  
      @Override
      public Integer nextNumber(int i) {
          int returnValue = 1;
          if (cards.indexOf(preValue) != -1)// 出すカードがあるか調べる
          {// 該当のカードが手持ちにあれば,そのカードを出す
              returnValue = preValue;
          } else {// 該当するカードがない時には,0 番目にあるカードを出す
              returnValue = cards.get(0);
          }
          int index = cards.indexOf(returnValue);
          cards.remove(index);// 出した数字を削除する.数字は0 番から格納されている.
          return returnValue;
      }
  
      @Override
      public void learning(int enemyNumber) {// 出したカードを記憶しておく
          preValue = enemyNumber;
      }
  }
\end{lstlisting}

strategyRandomクラスをリスト\ref{strategyRandom}に示す。

\begin{lstlisting}[caption=strategyRandomクラス,label=strategyRandom]
  import java.util.ArrayList;
  import java.util.Collections;
  import java.util.List;
  
  public class strategyRandom implements Strategy {
          // 数値を格納するリストをつくる
          private List<Integer> cards = new ArrayList<Integer>();
  
          strategyRandom() {
                  // リストに数値を順に入れていく
                  cards.add(1);
                  cards.add(2);
                  cards.add(3);
                  cards.add(4);
                  cards.add(5);
                  // コレクションのシャッフル機能を使ってシャッフルする
                  Collections.shuffle(cards);//
          }
  
          // リストから順番に数値を出す
          public Integer nextNumber(int i) {
                  return cards.get(i);
          }
  
          // 過去の出し手を利用しないので何も操作しない実装をする
          @Override
          public void learning(int enemy) {
          }
  }
\end{lstlisting}





\end{document}