\documentclass[a4paper,11pt,dvipdfmx]{jsarticle}


% 数式
\usepackage{amsmath,amsfonts}
\usepackage{bm}
\usepackage{physics}
\usepackage{mathtools}
% 画像
\usepackage[dvipdfmx]{graphicx}
\usepackage{circuitikz}
\usepackage{amsmath,amssymb}
\usepackage{siunitx}
\usepackage{float}
\usepackage{tikz}
\usepackage{askmaps}
\usepackage{multirow}
\usepackage{bigstrut}
\usepackage{rotating}
\usepackage{listings}
\usepackage{subcaption}
% 表
\usepackage{makecell}
% その他
\usepackage{url}
\usepackage{ascmac}
\usepackage{cases}
\usepackage{here}
\usepackage{upgreek}
\usepackage{tocloft}  % tocloftパッケージを使う
\usepackage{titlesec} % titlesecパッケージを使う(セクションタイトルのカスタマイズ)

% 画像挿入コマンド
\newcommand{\Figure}[4]{
\begin{figure}[H]
\centering
\includegraphics[width=#1\linewidth]{./images/#2}
\caption{#3}
\label{fig:#4}
\end{figure}
}
\begin{document}

\section{要旨}
本実験光電効果を利用してプランク定数を求める.赤,緑,青色の
光電効果で観測する.メートルブリッジを用いて,電圧ごとの
光電流の変化,阻止電圧を測定し電圧と光の振動数の関係をグラフ化し,
傾きからプランク定数,切片から仕事関数を求める.

\section{目的}
光電管を用いて光電効果の減少を理解し,プランク定数と
仕事関数を求めることを本実験の目的とする.

\section{実験方法}
実験指導書\cite{key1}に従い,実験を行う.
なお,メートル両端電圧は1.2[V]とする.

\section{実験結果}
この章では測定したデータ,およびグラフを示す.
\subsection{測定データ}
電気素量 \( e \) は \( 1.602 \times 10^{-19} \mathrm{C} \) とする.
赤,緑,青色の光によるメートルブリッジの値,阻止電圧,
検流計の値,および電子のエネルギーeV値の実行結果を
表\ref{tab:result}に示す.
阻止電圧[V]はメートルブリッジの両端電圧1.2[V]
に対する距離の比であるため式\ref{eq:voltage}で求める.
\begin{equation}
V = 測定距離[m] \times 両端電圧1.2[V]
\label{eq:voltage}
\end{equation}

\begin{table}[H]
\centering
\caption{各色におけるメートルブリッジの値と阻止電圧,検流計の値,および eV 値}
\label{tab:result}
\begin{tabular}{|c|c|c|c|c|}
\hline
色 & メートルブリッジ [m] & 阻止電圧 [V] & 検流計の値 & $eV \times 10^{19}$ [J] \\
\hline
\multirow{7}{*}{Red} 
& 0.010 & 0.012 & 250 & 0.0192 \\
& 0.100 & 0.120 & 110 & 0.192 \\
& 0.200 & 0.240 & 21  & 0.384 \\
& 0.230 & 0.276 & 12  & 0.442 \\
& 0.260 & 0.312 & 8   & 0.499 \\
& 0.300 & 0.360 & 2   & 0.577 \\
& 0.320 & 0.384 & 0   & 0.615 \\
\hline
\multirow{5}{*}{Green}
& 0.320 & 0.384 & 232 & 0.615 \\
& 0.420 & 0.504 & 40  & 0.807 \\
& 0.470 & 0.564 & 17  & 0.904 \\
& 0.520 & 0.624 & 9   & 0.999 \\
& 0.530 & 0.636 & 0   & 1.02 \\
\hline
\multirow{4}{*}{Blue}
& 0.640 & 0.768 & 245 & 1.23 \\
& 0.740 & 0.888 & 39  & 1.42 \\
& 0.790 & 0.948 & 10  & 1.52 \\
& 0.800 & 0.960 & 0   & 1.54 \\
\hline
\end{tabular}
\end{table}

検流計の値が0の時が限界阻止電圧である.この結果を
電流系の値-阻止電圧グラフとしてリスト1に示す.

\newpage

赤,緑,青色の光に対する限界阻止電圧はそれぞれ
0.615[V],1.02[V],1.54[V]である.このとき
の各色の振動数に対する電子のエネルギー
$eV_0$[J]を表\ref{tab:frequency}に示す.各色の
振動数$v$[Hz]は式\ref{eq:frequency}で求める.
\begin{equation}
v = \frac{c}{\lambda}
\label{eq:frequency}
\end{equation}

\begin{table}[H]
\centering
\caption{各色の振動数と電子のエネルギー}
\label{tab:frequency}
\begin{tabular}{|c|c|c|c|}
\hline
色 & 波長 $\lambda$ [nm] & 振動数 $\nu$ [Hz] & 電子のエネルギー $eV_0$ [J] \\
\hline
Red   & $635 \pm 10$ & $4.72 \times 10^{14}$ & $0.615 \times 10^{-19}$ \\
Green & $532 \pm 10$ & $5.64 \times 10^{14}$ & $1.02 \times 10^{-19}$ \\
Blue  & $450 \pm 10$ & $6.67 \times 10^{14}$ & $1.54 \times 10^{-19}$ \\
\hline
\end{tabular}
\end{table}

この結果から振動数 $\nu$-電子のエネルギー $eV_0$ のグラフを作成する.
$eV_0$ とプランク定数 $h$,振動数 $\nu$,仕事関数 $W$ の関係は式~(\ref{eq:photo}) のように表される.

\begin{equation}
eV_0 = h\nu - W
\label{eq:photo}
\end{equation}

実験結果から最小二乗法を用いて傾き $a$ と切片 $b$ を求める.傾き $a$ と切片 $b$ は次のように求められた.

\begin{equation}
a = 6.79 \times 10^{-34}, \quad b = -2.23 \times 10^{-19}
\end{equation}

相関係数 $r$ は 1 に近く,グラフは直線に近い.傾き $a$ はプランク定数を表し,切片 $b$ は仕事関数を表す.

よってプランク定数 $h$ と仕事関数 $W$ の値は次のように求められた.

\begin{equation}
h = 6.79 \times 10^{-34} \;[\mathrm{J \cdot s}], \quad 
W = -2.23 \times 10^{-19} \;\mathrm{[J]} = -1.39 \;\mathrm{[eV]}
\label{eq:result}
\end{equation}

\section{考察}
今回の実験では,光電効果を利用してプランク定数および仕事関数を求めた.
測定した赤,緑,青色の各光に対する限界阻止電圧と振動数の関係をグラフ化し,その直線の傾きからプランク定数を,切片から仕事関数を求めた.

得られたプランク定数の値は $h = 6.79 \times 10^{-34}\,\mathrm{J \cdot s}$ であり,理論値 $6.63 \times 10^{-34}\,\mathrm{J \cdot s}$ に対しておおよそ $2.4\%$ の誤差であった.この誤差は比較的小さく,実験結果は理論値と良く一致しているといえる.

一方で,仕事関数は $W = -2.23 \times 10^{-19}\,\mathrm{J} = -1.39\,\mathrm{eV}$ となった.これは一般的な金属の仕事関数と比べると若干大きい値である.これは,光源の波長に不確かさがあったこと,また阻止電圧の測定に主観的判断が含まれることなどが原因と考えられる.

特に阻止電圧の決定には検流計の値がゼロになる点を目視で判断する必要があり,わずかな誤差がグラフの傾きと切片に大きな影響を与える可能性がある.また,メートルブリッジの銅線が完全に直線でなかった場合,距離の測定精度が低下し,それに伴って阻止電圧の誤差が生じる可能性もある.

さらに,赤色光はエネルギーが低いため,他の光の影響を受けやすく,限界阻止電圧の測定が特に難しかった.実際,表\ref{tab:result}を見ると赤色光の電流値の変化が比較的緩やかであり,ゼロ点の特定が不明瞭であったことがうかがえる.

以上のことから,実験結果はおおむね理論と一致しており,光電効果の理解およびプランク定数の算出に成功したといえるが,阻止電圧測定に関する系統的誤差の影響を受けている可能性がある.今後の実験では,より高精度な測定方法や自動検出装置を導入することで精度の向上が期待される.




\begin{thebibliography}{9}
\bibitem{key1}
物理学実験指導書編集委員会編『新物理学実験』,学術図書出版社,2024年。
\end{thebibliography}
\end{document}